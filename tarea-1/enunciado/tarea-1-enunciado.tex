\documentclass[12pt,a4paper]{amsart}
\usepackage[utf8]{inputenc}
\usepackage[T1]{fontenc}
\usepackage[spanish]{babel}
\usepackage{amsmath}
\usepackage{amsfonts}
\usepackage{amssymb}
\usepackage{graphicx}
\author{Prof. Juliho Castillo Colmenares}
\title{Matemáticas para la Ciencia de Datos\\Tarea 1}
\date{Actualizada al \today}

\begin{document}
    \maketitle

Definamos $f(x)=\frac{1}{1+x}$.

    \begin{enumerate}
\item Calcula la integral exacta $\displaystyle \int_{0}^{1}f(x)dx.$
\item  Usa el método trapezoidal simple ($n=1$) para estimar la integral.
\item  Calcula el error relativo para el método trapezoidal simple.
\item Usa la regla trapezoidal compuesta con $n=2$ subintervalos para aproximar la integral.
\item Calcula el error relativo para la regla trapezoidal compuesta con $n=2$.
\item Usa la regla trapezoidal compuesta con $n=3$ subintervalos para aproximar la integral.
\item Calcula el error relativo para la regla trapezoidal compuesta con $n=3$.
\item Generaliza los resultados anteriores para estimar cuál debería ser el valor mínimo de $n$ para que el error relativo sea menor a $0.001$.
    \end{enumerate}

\end{document}